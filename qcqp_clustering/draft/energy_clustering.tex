\documentclass[aps,preprint,nofootinbib,floatfix]{revtex4-1}
%\documentclass{article}

%\usepackage{nips_2017}

% to compile a camera-ready version, add the [final] option, e.g.:
%\usepackage[final]{nips_2017}

%\usepackage{nicefrac}       % compact symbols for 1/2, etc.
%\usepackage{microtype}      % microtypography
\usepackage{amsmath,amssymb,amsfonts,amsthm,amscd,bm,bbm}
%\usepackage{mathtools}
%\usepackage{printlen}
%\usepackage[inline]{enumitem}
%\usepackage{cite}
%\usepackage{bbm}

\usepackage[pdftex]{graphicx}
\graphicspath{{./figs/}}

\usepackage{algorithmic}
\usepackage{algorithm}

\hyphenation{op-tical net-works semi-conduc-tor}

\newtheorem{theorem}{Theorem}
\newtheorem{definition}[theorem]{Definition}
\newtheorem{assumption}[theorem]{Assumption}
\newtheorem{lemma}[theorem]{Lemma}
\newtheorem{corollary}[theorem]{Corollary}
\newtheorem{proposition}[theorem]{Proposition}
\newtheorem{conjecture}[theorem]{Conjecture}
\newtheorem{remark}[theorem]{Remark}
\newtheorem{example}{Example}

%% our definitions %%%%%%%%%%%%%%%%%%%%%%%%%%%%%%%%%%%%%%%%%%%%%%%%%%%%%%%%%%%%
\DeclareMathOperator{\aff}{aff}
\DeclareMathOperator{\st}{s.t.}
\DeclareMathOperator{\LC}{LC}
\DeclareMathOperator{\affnot}{aff_0}
\DeclareMathOperator{\conv}{conv}
\DeclareMathOperator{\relint}{relint}
\DeclareMathOperator{\vol}{vol}
\DeclareMathOperator{\range}{range}
\DeclareMathOperator{\image}{im}
\DeclareMathOperator{\nullspace}{null}
\DeclareMathOperator{\area}{area}
\DeclareMathOperator{\vspan}{span}
\DeclareMathOperator{\id}{Id}
\DeclareMathOperator{\cond}{cond}
\DeclareMathOperator{\prox}{prox}
\DeclareMathOperator*{\argmax}{arg\,max}
\DeclareMathOperator*{\argmin}{arg\,min}
\DeclareMathOperator*{\minimize}{minimize}
\DeclareMathOperator{\diag}{diag}
\DeclareMathOperator{\Tr}{Tr}

\newcommand\Energy{\mathcal{E}}
\newcommand\E{\mathbb{E}}
\newcommand\kk{K}
\newcommand\kkk{h}
\newcommand\Hk{{\mathcal{H}}_{\kk}}
\newcommand\HH{\mathcal{H}}
\newcommand\C{{\mathcal{C}}}
\newcommand\OO{{\mathcal{O}}}
%\newcommand\Zt{\widetilde{Z}}
\newcommand\Zt{Y}
%\newcommand{\Ind}[1]{\delta_{#1}}
\newcommand{\Ind}[1]{\mathbbm{1}_{#1}}


\begin{document}

\title{Nonparametric Clustering from  Energy Statistics}

\author{Guilherme Fran\c ca}
\email{guifranca@gmail.com} 
%\affiliation{Johns Hopkins University}

\author{Joshua T. Vogelstein}
\email{jovo@jhu.edu}

\affiliation{Johns Hopkins University}


\begin{abstract}
Energy statistics provides a nonparametric test for equality of distributions.
It was proposed by 
Sz\' ekely in the 80's
inspired by the Newtonian gravitational potential from classical mechanics. 
The idea
is to associate a statistical potential energy to observations, such that 
minimum energy is achieved under the null hypothesis of equality of 
distributions. Energy statistics
was further generalized to probability 
distributions on arbitrary metric spaces,
and more recently, a connection with kernels in RKHS was established.
Nevertheless, although extensively used by the statistics community, energy
statistics has not been
previously incorporated in machine learning problems.
In this paper, we consider the problem of clustering data from
an energy statistics theory perspective.
We provide a precise mathematical formulation, yielding
a quadratically constrained quadratic optimization problem (QCQP). 
We show that
this is equivalent to kernel $k$-means
optimization problem, however,
energy statistics is able to fix the kernel choice. 
This clustering 
method is nonparametric, and if prior information is available
it can be easily incorporated in the kernel construction.
We propose an iterative algorithm
to find local optimizers of this QCQP problem, based on the change
in the statistical energy 
of moving points to different clusters. This algorithm 
is different but has the same computational cost as kernel $k$-means 
algorithm.
We then compare this algorithm with
well-known kernel $k$-means, standard $k$-means, and GMM algorithms by 
providing carefully designed numerical experiments. The results
show that, in general, this method outperforms
these most used clustering algorithms.
\end{abstract}

\maketitle

\section{Introduction}

Energy statistics is based on the energy distance between probability
distributions, which provides a notion of statistical potential energy
to statistical observations, in close analogy to Newton's gravitational
potential energy in classical mechanics. We refer the reader to
\cite{Szkely2013}, and references therein, for an overview.
It has been applied to several goodness-of-fit hypothesis tests, 
multi-sample tests of equality of distributions, analysis of variance
\cite{RizzoVariance}, and (nonlinear)
dependence tests through
distance covariance and distance correlation, which generalizes the Pearson
correlation coefficient. 
Moreover, energy statistics  was applied
to hierarchical clustering \cite{RizzoClustering} by extending Ward's
method of minimum variance.

More recently, the distance covariance was 
generalized from Euclidean
spaces to metric spaces of strong negative type \cite{Lyons}.
Furthermore, a unifying framework establishing an equivalence
between generalized energy distances to maximum
mean discrepancies (MMD), which are distances between embeddings of
distributions in reproducing kernel Hilbert spaces (RKHS), was
provided \cite{Sejdinovic2013}. This important work establishes the link
between techniques commonly used in the statistics literature, regarding
energy statistics, and techniques
commonly used in machine learning.

Given a dataset $\mathbb{X} = \{x_1, x_2,\dotsc,x_n \}$, where each
datapoint lives in a generic space $x_i \in \mathcal{X}$, the $k$-clustering
problem consists in grouping these points into $k$ groups $\{
\C_1,\C_2,\dotsc,\C_k \}$, such that
points belonging to the same group are more ``similar'' to each other
than to points in  other groups. This intuitive notion already assumes
a ``metric'' able to measure the similarity between datapoints. Clustering
is an unsupervised method, and one of the most important problems in machine
learning since it provides the first step towards automatically constructing
labels for previously unseen data. The most used algorithms are
by far $k$-means and gaussian mixture models (GMM) through expectation
maximization (EM) algorithm. Both are parametric, making 
strong assumptions about
the distribution of the data (normality), and the 
space where data lie is Euclidean, $\mathcal{X} = \mathbb{R}^D$, hence
the similarity is based on Euclidean metric, 
$\| x_i - x_j \|^2 = (x_i - x_j)^\top(x_i-x_j) = 
\sum_{\ell=1}^D (x_{i,\ell} - x_{j,\ell})^2$. These algorithms provide
a good clustering when data is linearly separable in Euclidean space.

To account for nonlinearly separable data, which may possibly live
in an arbitrary non-Euclidean space, kernel methods are usually employed.
If a Mercer kernel $\kk: \mathcal{X}\times \mathcal{X} \to \mathbb{R}$ 
is available \cite{Mercer}, which is a symmetric and positive definite 
function, it guarantees that there exists a map $\varphi : \mathcal{X} \to
\mathcal{H}$, where $\mathcal{H}$ is a Hilbert space. Thus, one can
compute similarities through the inner product and 
perform clustering in the feature space $\mathcal{H}$. The nice thing
about this is that the map $\varphi$ is only used implicitly, i.e. one
does not need to know $\varphi$, since the inner
product can be computed only based on the kernel $\kk(\cdot, \cdot)$, 
a convenient technique known as kernel trick. The well-known 
kernel $k$-means problem 
is exactly $k$-means in the feature space
\cite{Smola,Girolami} (see also \cite{Filippone} for a survey of clustering
methods). 
It exploits the nonlinear structure provided by
a kernel function to perform clustering in datasets that are not linearly
separable in their original space $\mathcal{X}$.
As it stands, kernel $k$-means is an heuristic approach in the sense that
it is not derived from a statistical theory. Moreover, the choice of
kernel is crucial, and there is no systematic theory for this. One
must rely on ad hoc methods.

In this paper, we consider how to perform clustering based on energy
statistics theory. The original goal is to provide a nonparametric
clustering method, since energy distance is nonparametric.
The basis of our work is the theory developed in 
\cite{Sejdinovic2013}. Based on this and on the multi-sample test
for equality of distributions from energy statistics, we show that
there is only one function that provides the clustering optimization
problem. We then show that this clustering
optimization problem reduces to a quadratically constrained quadratic
problem (QCQP). This optimization problem has the same form as kernel $k$-means,
spectral clustering, and some well-known graph partitioning problems
\cite{Dhillon}. This lead us to show that energy statistics based
clustering is actually equivalent to kernel $k$-means optimization problem.
However, energy statistics fixes a kernel to begin with.
Furthermore, these results can be seen as a first-principle derivation
of kernel $k$-means from energy statistics, which thus bring
these kernel based clustering methods into a unifying mathematical theory.
We also provide an iterative algorithm, different but with the same
complexity as kernel $k$-means algorithm. The numerical results indicate
that this method is more accurate and robust.

Our work is organized as follows. In section~\ref{sec:background} we review
the necessary background on energy statistics and RKHS.
Section~\ref{sec:clustering_theory} contains the main results of this paper
where we consider a clustering theory based on energy statistics, leading
to a QCQP which is NP-hard.
In section~\ref{sec:twoclass} we consider a simple example in one dimension,
where we propose an algorithm which requires no initialization.
In section~\ref{sec:algo} we briefly review kernel $k$-means algorithm,
and propose a new iterative algorithm to solve this QCQP.
Section~\ref{sec:numerics} contains some carefully designed numerical
experiments indicating that this algorithm outperforms kernel
$k$-means, standard $k$-means, and GMM algorithms.
Our conclusions are in section~\ref{sec:conclusion}.



%%%%%%%%%%%%%%%%%%%%%%%%%%%%%%%%%%%%%%%%%%%%%%%%%%%%%%%%%%%%%%%%%%%%%%%%%%%%%%%
\section{Background on Energy Statistics and RKHS}
\label{sec:background}

In this section we briefly review the main concepts from energy
statistics and its relation to reproducing kernel Hilbert spaces 
(RKHS) which form the basis of our work.
For more details we refer the reader
to \cite{Szkely2013} (and references therein) and 
also \cite{Sejdinovic2013}.

Consider random variables in $\mathbb{R}^D$ 
such that $X,X' \stackrel{iid}{\sim} P$ and 
$Y,Y' \stackrel{iid}{\sim} Q$, where $P$ and $Q$ are cumulative
distribution functions with finite first moments. 
The quantity \cite{Szkely2013}
\begin{equation}
\label{eq:energy}
\Energy(P, Q) \equiv 2 \E \| X - Y\| - \E \| X - X' \| - \E \| Y - Y' \|,
\end{equation}
called \emph{energy distance}, 
is rotational invariant and nonnegative, $\Energy(P,Q) \ge 0$, where
equality
to zero holds if and only if $P = Q$.
Above $\| \cdot \|$ denotes the
Euclidean norm in $\mathbb{R}^D$. 
Energy distance
provides a characterization of equality of distributions, and
$\Energy^{1/2}$ is
a metric on the space of distributions.

The energy distance can be generalized as, for instance,
\begin{equation}
\label{eq:energy2}
\Energy_\alpha(P, Q) \equiv 
2 \E \| X - Y\|^{\alpha} - \E \| X - X' \|^{\alpha} - 
\E \| Y - Y' \|^{\alpha},
\end{equation}
where $0<\alpha\le 2$. This quantity is also nonnegative,
$\Energy_\alpha(P,Q) \ge 0$. Furthermore, for $0<\alpha<2$ we have that
$\Energy_\alpha(P,Q) = 0$ if and only if $P=Q$, while for $\alpha=2$ 
we have $\Energy_2(P,Q) = 2\| \E(X) - \E(Y) \|^2$, showing that
equality to zero only requires
equality of the means, and thus $\Energy_2(P,Q)=0$ does 
not imply equality of distributions.

It is important to 
mention that \eqref{eq:energy2} can be even further generalized.
Let $X, Y \in \mathcal{X}$,  where $\mathcal{X}$ is a space endowed
with a \emph{semimetric of negative type}
$\rho: \mathcal{X}\times\mathcal{X} \to \mathbb{R}$, which is required
to satisfy
\begin{equation}
\label{eq:negative_type}
\sum_{i,j=1}^n \alpha_i \alpha_j \rho(X_i, X_j) \le 0,
\end{equation}
where $X_i \in \mathcal{X}$ and $\alpha_i \in \mathbb{R}$ such that
$\sum_{i=1}^n \alpha_i = 0$. Then, $\mathcal{X}$ is called a \emph{space of
negative type}.
We can thus replace $\mathbb{R}^D \to \mathcal{X}$ and 
$\| X - Y \| \to \rho(X , Y)$ in the definition \eqref{eq:energy}, obtaining
the energy distance as
\begin{equation}
\label{eq:energy3}
\Energy(P, Q) \equiv 2 \E \rho(X,Y) - \E \rho(X, X') - \E \rho(Y,Y').
\end{equation}
For spaces of negative type, there exists a Hilbert space $\mathcal{H}$ and
a map $\varphi: \mathcal{X} \to
\mathcal{H}$ such that
$\rho(X, Y) = \| \varphi(X) - \varphi(Y) \|_{\mathcal{H}}^2$, which
allows us to compute quantities related to probability distributions over
$\mathcal{X}$ in the Hilbert space $\mathcal{H}$.
Even though the semimetric 
$\rho$ may not satisfy the triangle inequality, 
$\rho^{1/2}$ does since it can be shown to be a legit metric. 

There is an equivalence between energy distance, 
commonly used in statistics,
and distances between embeddings of distributions in 
RKHS, commonly used in machine learning. 
This equivalence was established
in \cite{Sejdinovic2013}. We first recall the definition of
RKHS. Let $\HH$ be a Hilbert space of real-valued functions
over $\mathcal{X}$. A function 
$\kk : \mathcal{X} \times \mathcal{X} \to 
\mathbb{R}$ is a reproducing kernel of $\HH$ if it satisfies
the following two conditions:
\begin{enumerate}
\item $\kkk_x \equiv \kk(\cdot, x) \in \HH$ 
for all $x \in \mathcal{X}$.
\item $\langle \kkk_x, f \rangle_{\HH} = f(x)$ for
all $x\in\mathcal{X}$ and $f\in \HH$.
\end{enumerate}
In other words, for any $x \in \mathcal{X}$ and any function $f \in \HH$,
there is a unique 
$\kkk_x \in \HH$ that reproduces $f(x)$ through the inner product
of $\HH$.
If such a \emph{kernel} 
function $\kk$ exists, then $\HH$ is called a RKHS. The above two 
properties immediately imply that $\kk$ is symmetric and positive
definite. Indeed, notice that
$\langle \kkk_x, \kkk_y \rangle = \kkk_y(x) = \kk(x,y)$, and since
this inner product is real,
$\langle \kkk_x, \kkk_y \rangle^* = \langle \kkk_y, \kkk_x \rangle = 
\langle \kkk_x, \kkk_y \rangle$, we immediately have that
the kernel is symmetric,
$\kk(y,x) = \kk(y,x)$. Moreover, for any $w \in
\HH$ we can write $w = \sum_{i=1}^n c_i \kkk_{x_i}$, where
$\{ \kkk_{x_i} \}_{i=1}^n$ is a basis of $\HH$. It follows that
$\langle w, w \rangle_{\HH}  = \sum_{i,j=1}^n c_i c_j \kk(x_i,x_j) \ge 0$,
showing that the kernel is positive definite. If $G$ is a matrix with
elements $G_{ij} = \kk(x_i,x_j)$, this is equivalent to $G$ being
positive semi-definite: $\bm{v}^\top G \, \bm{v} \ge 0$ for any vector
$\bm{v} \in \mathbb{R}^n$.

The Moore-Aronszajn theorem 
\cite{Aronszajn}
establishes the converse of the above paragraph. 
For every symmetric
and positive definite function $\kk: \mathcal{X}\times \mathcal{X} \to
\mathbb{R}$, there is an associated RKHS $\Hk$ 
with reproducing
kernel $\kk$. The map $\varphi: x \mapsto \kkk_x \in \Hk$ is called
the canonical \emph{feature map}. Given a kernel $\kk$,
this theorem enables us to define an embedding of a probability measure
$P$ into the RKHS: $P \mapsto \kkk_P \in
\Hk$ such that 
$\int f(x) d P(x) = \langle f, \kkk_P \rangle$ for all $f \in \Hk$,
or alternatively, $\kkk_P \equiv \int \kk( \, \cdot \,, x)  d P(x)$. 
We can now  introduce the 
notion of distance between two probability measures using the inner product
of $\Hk$. This is called the maximum mean discrepancy (MMD) and
is given by
\begin{equation}\label{eq:mmd}
\gamma_\kk(P,Q) \equiv \| \kkk_P - \kkk_Q \|_{\Hk},
\end{equation}
which can also be written as \cite{Gretton2012}
\begin{equation}\label{eq:mmd2}
\gamma_\kk^2(P,Q) = \E \kk(X,X') + \E \kk(Y,Y') - 2 \E \kk(X, Y)
\end{equation}
where $X,X' \stackrel{iid}{\sim} P$ and $Y,Y'\stackrel{iid}{\sim} Q$.
From the equality between \eqref{eq:mmd} and \eqref{eq:mmd2} we also
have 
\begin{equation}\label{eq:inner_data}
\langle \kkk_P, \kkk_Q \rangle_{\Hk} = \E \, \kk(X, Y).
\end{equation}
Thus, in practice, we can estimate the inner product between the 
embedded distributions 
by averaging the kernel function over sampled data.

The following important result shows that semimetrics of negative
type and symmetric positive semidefinite kernels are closely related
\cite{Berg1984}. Let $\rho: \mathcal{X} \times \mathcal{X} \to \mathbb{R}$,
and $x_0 \in \mathcal{X}$ an arbitrary but fixed point.
Define
\begin{equation}
\label{eq:kernel_semimetric}
\kk(x,y) = \tfrac{1}{2} \left[  \rho(x,x_0) + \rho(y,x_0) - \rho(x,y)\right].
\end{equation}
Thus, it can be shown that 
$\kk$ is positive definite if and only if $\rho$ is a semimetric
of negative type
\eqref{eq:negative_type}.
Here we have a family of kernels, one for each choice of $x_0$. Conversely,
if $\rho$ is a semimetric of negative type and $\kk$ is a kernel in this
family, then 
\begin{equation}
\label{eq:gen_kernel}
\begin{split}
\rho(x,y) &= \kk(x,x) + \kk(y,y) -2\kk(x,y) \\
&=  \| \kkk_x - \kkk_y \|^2_{\Hk},
\end{split}
\end{equation}
and the canonical feature map 
$\varphi: x \mapsto \kkk_x$ is injective \cite{Sejdinovic2013}.
When these conditions are satisfied, we say that the kernel $\kk$ 
generates the semimetric $\rho$. 
If two different kernels generate the same $\rho$ they are
equivalent kernels.

Now we can state the equivalence between energy distance $\Energy$ and
inner products on RKHS, which is one of the main results of
\cite{Sejdinovic2013}. If $\rho$ is a semimetric
of negative type and $\kk$ a kernel that generates $\rho$, then
replacing \eqref{eq:gen_kernel} into
\eqref{eq:energy3}, and using \eqref{eq:mmd2}, yields
\begin{equation} \label{eq:Erho}
\Energy(P, Q) = 
2 \left[ \E \, \kk(X, X') + \E \, \kk(Y, Y') - 2\E \, \kk(X, Y)\right] 
= 2 \gamma_\kk^2(P,Q) .
\end{equation}
Since $\gamma_k^2(P, Q) = \| \kkk_P - \kkk_Q \|^2_{\Hk}$ we
can compute the energy distance using the inner product of $\Hk$. Moreover,
this can be computed from the data according to \eqref{eq:inner_data}.

Finally, let us recall the main formulas for test statistics
of equality of distributions \cite{Szkely2013}. 
Assume we have data $\mathbb{X} = \{ x_1,\dotsc, x_n \}$, where
$x_i \in \mathcal{X}$ and $\mathcal{X}$ is a space of negative type.
Consider a partition $\mathbb{X} = \bigcup_{j=1}^k \C_j$, with
$\C_i \cap \C_j = \emptyset$.
Each expectation in 
\eqref{eq:energy3}
can be computed 
through the function
\begin{equation}
\label{eq:g_def}
g (\C_i, \C_j) \equiv 
\dfrac{1}{n_i n_j}
\sum_{x \in \C_i} 
\sum_{y \in \C_j} \rho(x, y)
\end{equation}
where $n_i = |\C_i|$ is the number of elements in 
$\C_i$. 
The \emph{within energy dispersion} is defined by
\begin{equation}
\label{eq:within}
W \equiv
\sum_{j=1}^{k} \dfrac{n_j}{2} g(\C_j, \C_j),
\end{equation}
and the \emph{between-sample energy statistic} is defined by
\begin{equation}
\label{eq:between}
S \equiv
\sum_{1 \le  i < j \le k } \dfrac{n_i n_{j}}{2 n} \left[
2 g(\C_i, \C_j) - 
g(\C_i, \C_i) - 
g(\C_j, \C_j)
\right].
\end{equation}
Given a set of distributions
$\{ P_j\}_{j=1}^k$, where $x \in \C_j$ if and only if $x \sim P_j$, 
the quantity $S$ provides
a \emph{nonparametric} test statistic for equality of distributions
\cite{Szkely2013}.
When the sample size is large enough, $n\to \infty$,
under the null hypothesis $H_0: P_1=P_2=\dotsm=P_k$ we have that
$S\to 0$, 
and under
the alternative hypothesis $H_1: P_i \ne P_j$ for at least two $i\ne j$, 
we have that $S \to \infty$.
This test is nonparametric in the sense that it does not make any assumptions
about the distributions $P_j$.

One can make the analogy 
that every data point $ x \in \C_j$ form a massive body, 
whose total mass is characterized by the distribution function $P_j$. 
Then $S$ is a potential
energy of the from $S(P_1,\dotsc,P_k)$ which measures how different
are these
mass distributions, and achieves the ground state
$S=0$ when all bodies have the same mass distribution. The potential energy
$S$ increases as each body has different mass distribution than the other 
ones.



%%%%%%%%%%%%%%%%%%%%%%%%%%%%%%%%%%%%%%%%%%%%%%%%%%%%%%%%%%%%%%%%%%%%%%%%%%%%%%%
\section{Clustering Based on Energy Statistics}
\label{sec:clustering_theory}

This section contains the main results of this paper, where 
we formulate an optimization problem for clustering 
based on energy statistics and RKHS introduced in the previous section.

Due to the test statistic \eqref{eq:between} for equality of distributions,
the obvious
criterion for clustering data is to 
maximize $S$, which makes 
each cluster as different
as possible from the other ones.
In other words, given a set of points coming from different probability
distributions, $S$ should attain a maximum when each point is correctly
classified as belonging to the cluster associated to its probability
The following 
straightforward result
shows that maximizing \eqref{eq:between} is equivalent to minimizing
\eqref{eq:within}, which has a more convenient form.

\begin{proposition}
\label{th:minimize}
Let $\mathbb{X} = \{x_1,\dotsc,x_n\}$ where each data point
$x_i$ lives in a space $\mathcal{X}$ endowed with a semimetric $\rho:
\mathcal{X}\times\mathcal{X} \to \mathbb{R}$ of
negative type \eqref{eq:negative_type}. For a fixed integer $k$,
the partition
$\mathbb{X} = \bigcup_{j=1}^k \C_j$, where $\C_i \cap C_j = \emptyset$ for
all $i\ne j$, maximizes \eqref{eq:between} if and only if
\begin{equation}
\label{eq:minimize}
\min_{\C_1,\dotsc,C_k  } W(
\C_1, \dotsc, \C_k),
\end{equation}
where $W$ is given by \eqref{eq:within}.
\end{proposition}
\begin{proof}
From \eqref{eq:within} and \eqref{eq:between}, and recall
that $n=\sum_{i=1}^k n_i$, we have
\begin{equation}
\begin{split}
S + W &= 
\dfrac{1}{2n} \sum_{\substack{i,j=1 \\ i\ne j}}^k n_i n_j g(\C_i, \C_j)
+ \dfrac{1}{2n} \sum_{i=1}^{k} n_i g(\C_i, \C_i) \left( n - 
\sum_{j\ne i = 1}^k n_j \right) \\
& = \dfrac{1}{2n} \sum_{i,j=1}^k n_i n_j g(\C_i, \C_j)
= \dfrac{1}{2n} \sum_{x \in \mathbb{X}} \sum_{y \in \mathbb{X}} \rho(x,y)
= \dfrac{n}{2} g(\mathbb{X}, \mathbb{X}).
\end{split}
\end{equation}
Note that the right hand side of this equation 
only depends on the pooled data, so it is a constant
independent of the choice of partition. Therefore, maximizing
$S$ over the choice of partition is equivalent to minimizing~$W$.
\end{proof}

Thus, for a given $k$, the clustering problem amounts to
finding the best partition of the data by solving~\eqref{eq:minimize}.
Notice that this is a hard assignment clustering problem.

Now we show how to formulate problem \eqref{eq:minimize} in the corresponding
RKHS.
Based on 
\eqref{eq:kernel_semimetric} and \eqref{eq:gen_kernel}, assume that 
the kernel $\kk: \mathcal{X} \times \mathcal{X} \to \mathbb{R}$ 
generates $\rho$. 
Let us define  the Gram matrix
\begin{equation}
\label{eq:kernel_matrix}
G \equiv \begin{pmatrix}
\kk(x_1,x_1) & \kk(x_1,x_2) & \dotsm & \kk(x_1,x_n) \\
\kk(x_2,x_1) & \kk(x_2,x_2) & \dotsm & \kk(x_2,x_n) \\
\vdots & \vdots & \ddots  & \vdots \\
\kk(x_n,x_1) & \kk(x_n,x_2) & \dotsm & \kk(x_n,x_n) 
\end{pmatrix} .
\end{equation}
Let $Z \in \{ 0,1 \}^{n\times k}$ be the label matrix, 
with only one nonvanishing entry per row, 
indicating to which cluster (column)
each point (row) belongs to. This matrix satisfy
$Z^\top Z = D$ where $D = \diag( n_1,\dotsc, n_k )$  contains
the number of points in each cluster. We also introduce the rescaled
matrix  $Y \equiv Z D^{-1/2}$. In component form they are given by
\begin{equation}
\label{eq:label_matrix}
Z_{ij} \equiv \begin{cases}
1 & \mbox{if $x_i \in \C_j$ } \\
0 & \mbox{otherwise}
\end{cases} \qquad
\Zt_{ij} \equiv \begin{cases}
\tfrac{1}{\sqrt{n_j}} & \mbox{if $x_i \in \C_j$ } \\
0 & \mbox{otherwise}
\end{cases} .
\end{equation}
Throughout the paper, we use the notation $M_{i\bullet}$ to denote
the $i$th row of a matrix $M$, and $M_{\bullet j}$ denotes its $j$th column.
Our next result reveals the optimization problem behind \eqref{eq:minimize},
which is NP-hard since
it is a quadratically constrained quadratic problem (QCQP).

\begin{proposition} 
\label{th:qcqp2}
The problem \eqref{eq:minimize} is equivalent to
\begin{equation}
\label{eq:qcqp2}
\max_{\Zt} \Tr \left( \Zt^\top G \, \Zt \right)  \qquad
\mbox{s.t. $\Zt \ge 0$, $\Zt^\top \Zt = I$, 
$\Zt \Zt^\top \bm{e} = \bm{e}$},
\end{equation}
where $\bm{e} = (1,1,\dots,1)^\top \in \mathbb{R}^n$ is the all-ones vector,
and $G$ is the Gram matrix \eqref{eq:kernel_matrix}.
\end{proposition}
\begin{proof}
From 
\eqref{eq:gen_kernel},
\eqref{eq:g_def}, and
\eqref{eq:within}
we have
\begin{equation}
\label{eq:W2}
W(\C_1,\dotsc,\C_k  )
= \dfrac{1}{2} \sum_{j=1}^k \dfrac{1}{n_j} \sum_{x,y \in \C_j} \rho(x,y)
= \sum_{j=1}^k \sum_{x \in \C_j}  \bigg(
\kk(x,x) - \dfrac{1}{n_j} \sum_{y \in \C_j} \kk(x,y) \bigg).
\end{equation}
Note that the first term does not contribute to the optimization problem,
since it is a global term that does not depend which partition is chosen. 
Therefore, minimizing \eqref{eq:W2} is equivalent to
\begin{equation}
\label{eq:max_prob}
\max_{ \C_1,\dotsc,\C_k } 
\sum_{j=1}^k \dfrac{1}{n_j} \sum_{x,y\in C_j} \kk(x,y) .
\end{equation}
But 
\begin{equation}
\sum_{x, y \in \C_j} \kk(x, y) =
\sum_{p=1}^{n} \sum_{q=1}^{n} Z_{pj} Z_{qj} G_{pq} = 
(Z^\top G \, Z)_{jj},
\end{equation}
where we used the definitions \eqref{eq:kernel_matrix} and
\eqref{eq:label_matrix}. Thus the objective function in 
\eqref{eq:max_prob} is equal to
$\Tr \left( D^{-1} Z^\top G Z \right)$. Now we can
use the cyclic property
of the trace, and by the own definition of the matrix $Z$
in \eqref{eq:label_matrix} we obtain the following integer
programing problem:
\begin{equation}\label{eq:qcqp}
\max_{Z} \Tr\Big( \big( Z D^{-1/2}\big)^\top G 
\big( ZD^{-1/2} \big) 
\Big) \quad
\mbox{s.t. $Z_{ij} \in \{0,1\}$, $\sum_{j=1}^k Z_{ij} = 1$, 
$\sum_{i=1}^n Z_{ij} = n_j$.}
\end{equation}
Now we write this in terms of the matrix $Y = Z D^{-1/2}$.
The objective function immediately becomes
$\Tr\left( Y^\top G \, Y\right)$. Notice that the above constraints
imply that $Z^T Z = D$, where $D=\diag(n_1,\dotsc,n_k)$, which in turn gives
$D^{-1/2} Y^T Y D^{-1/2} = D$, or $Y^\top Y = I$. 
Also, every entry of $Y$ is positive by definition,
$Y \ge 0$. Now it only remains to show the last 
constraint in \eqref{eq:qcqp2}, which comes from the last
constraint in \eqref{eq:qcqp}. In matrix form this reads
$Z^T \bm{e} = D \bm{e}$. Replacing $Z=YD^{1/2}$ we have
$Y^\top \bm{e} = D^{1/2} \bm{e}$. Multiplying this last equation
on the left by $Y$, and noticing
that $Y D^{1/2} \bm{e} = Z \bm{e} = \bm{e}$, we finally obtain
$Y Y^\top \bm{e} = \bm{e}$. Thus, the optimization 
problem \eqref{eq:qcqp} is equivalent
to \eqref{eq:qcqp2} .
\end{proof}

Based on Proposition~\ref{th:qcqp2}, to group data $\{ x_1,\dotsc,x_n \}$
into  $k$ clusters, we first compute the Gram matrix
$G$ and then 
solve the optimization problem \eqref{eq:qcqp2} for $\Zt \in
\mathbb{R}^{n\times k}$. The $i$th row
of $\Zt$ will contain a single nonzero element in some $j$th column,
indicating that $x_i \in \C_j$. 
Problem \eqref{eq:qcqp2} is NP-hard and there
are few methods
available to solve it directly,
which is computational prohibitive even for small datasets.
However, one can find approximate solutions by relaxing some 
of the constraints, or obtaining a relaxed SDP version of the problem. 
For instance, the relaxed problem
\begin{equation}
\max_{Y} \Tr \left( Y^\top G \, Y \right) \quad \mbox{s.t. $Y^\top Y = I$}
\end{equation}
has a well-known closed form solution given by $Y^\star = U R$, where the
columns of $U$ contain the leading $k$ eigenvectors of $G$ corresponding
to the $k$ largest eigenvalues $\{ \lambda_1,\dotsc,\lambda_k \}$, and
$R \in \mathbb{R}^{k\times k}$ is an arbitrary orthogonal matrix. 
The resulting
optimal objective function is thus given by
$\max \Tr \left( {Y^\star}^\top G \, Y^\star \right)  = 
\sum_{i=1}^k \lambda_i$. One might then normalize and threshold the rows
of $Y^\star$, or even better, following \cite{NgJordan} we can normalize the
rows of $Y$ and apply standard $k$-means on this matrix where each
row is considered as a datapoint.
This same 
procedure usually done in spectral clustering on the (normalized) Laplacian
of the graph defined by a similarity matrix.
However, computing eigenvectors of a very large matrix
can be problematic, and usually iterative methods are preferred.

It is important to note 
that the previous theory for clustering based on energy statistics
hold for data living in an \emph{arbitrary space of negative type}.
This clustering method is
\emph{nonparametric} since it does not make any assumptions
about 
the distribution of the data,
contrary to $k$-means and gaussian mixture models (GMM), for example.
Moreover, this approach \emph{does not require} the concept of the 
\emph{cluster mean}
which can be ill-defined for some types of data, such as images for
instance, and thus it should also be robust to outliers.
This method is quite general and makes very few
assumptions about the data. If one uses the traditional energy distance
\eqref{eq:energy}, then $\rho(x,y) = \| x - y\|$ and this fixes the kernel
choice through \eqref{eq:kernel_semimetric}.
In practice, however,
the clustering quality strongly depend on the choice of a suitable
$\rho$ which is what measures the similarity between different data points,
and is equivalent to choosing an appropriate kernel.
Nevertheless, if prior knowledge is available for choosing $\rho$, 
or equivalently $K$,
this can easily be taken into account.

One may wonder how energy statistics clustering 
relates to the well-known kernel $k$-means problem.
We now address this question.
For a positive semidefinite $G$, there exists a map
$\varphi: \mathcal{X} \to \HH_\kk$ such that
$\kk(x,y) = \varphi(x)^\top \varphi(y)$. The kernel $k$-means optimization
problem,
in feature space,
is defined by
\begin{equation}
\label{eq:kernel_kmeans}
\min_{\C_1,\dotsc,\C_k}\bigg\{ 
J(\C_1,\dots,\C_k) \equiv  \sum_{j=1}^k
\sum_{x \in \C_j} \| \varphi(x) - \varphi(\mu_j) \|^2
\bigg\},
\end{equation}
where $\mu_j = \tfrac{1}{n_j} \sum_{x \in \C_j} x$ is the  mean of cluster
$\C_j$ in the ambient space.
It is known \cite{Dhillon} that problem \eqref{eq:kernel_kmeans} 
is equivalent to a QCQP in the same form as
\eqref{eq:qcqp2}. The next result makes this explicit, showing that
\eqref{eq:minimize} and \eqref{eq:kernel_kmeans} are actually equivalent,
once the kernel is fixed.

\begin{proposition}
\label{th:kernel_kmeans}
The clustering problem
\eqref{eq:minimize} based on energy statistics 
is equivalent to the kernel $k$-means problem
\eqref{eq:kernel_kmeans}, and both are equivalent to \eqref{eq:qcqp2}.
\end{proposition}
\begin{proof}
Notice that $\| \varphi(x) - \varphi(\mu_j) \|^2 = \varphi(x)^\top \varphi(x)
- 2 \varphi(x)^\top \varphi(\mu_j) + \varphi(\mu_j)^\top \varphi(\mu_j)$,
therefore
\begin{equation}
\label{eq:J}
J = \sum_{j=1}^k \sum_{x\in\C_j} \bigg(
\kk(x,x) - 
\dfrac{2}{n_j} \sum_{y\in \C_j} \kk(x,y) + \dfrac{1}{n_j^2}
\sum_{y,z \in \C_j} \kk(y,z) \bigg).
\end{equation}
The first term is global so it does not contribute to the optimization
problem. Notice that the third term gives
$\sum_{x\in\C_j} \tfrac{1}{n_j^2} \sum_{y,z\in\C_j} \kk(y,z) =
\tfrac{1}{n_j}\sum_{y,z\in\C_j} \kk(y,z)$, which is the same as
the second term. Thus the kernel $k$-means optimization problem is
\begin{equation}
\min_{\C_1,\dotsc,\C_k} J(\C_1,\dotsc,\C_k) = \max_{\C_1,\dotsc,\C_k}
\sum_{j=1}^k \dfrac{1}{n_j} \sum_{x,y \in\C_j} \kk(x,y)
\end{equation}
which is exactly the same as 
\eqref{eq:max_prob} from the energy statistics formulation. Therefore,
once the kernel $\kk$ is fixed, the function 
$W$ given by \eqref{eq:within} is the same
as $J$ in \eqref{eq:kernel_kmeans}.
The remaining of the proof proceeds as 
already shown in the proof of Proposition~\ref{th:qcqp2}, leading to
the optimization problem in the form \eqref{eq:qcqp2}.
\end{proof}

It was shown \cite{Dhillon} that kernel $k$-means, spectral clustering,
and graph partitioning problems such as ratio association, ratio cut, and
normalized cut are all equivalent to a QCQP of the form \eqref{eq:qcqp2}.
Actually, in general, 
this corresponds to a weighted version of \eqref{eq:qcqp2} which reads
$ \Tr \left( Y^\top W^{1/2} G \, W^{1/2} Y \right)$, where 
$W = 
\diag(w(x_1),\dotsc, w(x_n))$ and $w(\cdot)$ is a
weight attributed to each data point. Our previous 
results show that energy statistics
based clustering is also equivalent to these problems. The advantage
of energy statistics compared to kernel $k$-means 
is that the semimetric $\rho$ is supposed to be fixed. Moreover, from a
theoretical perspective, it brings the clustering problem into a formal
statistical theory based on distances between probability distributions
and embeddings in RKHS. There was no a priori reason to expect that clustering
based on energy statistics would be equivalent to the kernel $k$-means
problem~\eqref{eq:kernel_kmeans}.


%%%%%%%%%%%%%%%%%%%%%%%%%%%%%%%%%%%%%%%%%%%%%%%%%%%%%%%%%%%%%%%%%%%%%%%%%%%%%%%
\section{Two-Class Problem in One Dimension}
\label{sec:twoclass}

Before stating a general algorithm to solve \eqref{eq:qcqp2}, 
let us first consider the simplest possible case which
is one-dimensional data and a two-class problem. This will also 
be useful later
for comparison with the more general iterative algorithm.

Fixing
$\rho(x,y) = |x - y|$ according to \eqref{eq:energy}, we can actually compute 
\eqref{eq:g_def} in $\OO(n \log n)$ and find
a direct solution to \eqref{eq:minimize}. 
This is done by noticing that
\begin{equation}
\begin{aligned}
|x - y|  &= (x-y)\Ind{x \ge y} -
(x-y) \Ind{x < y}  \\
&= 
x \left( \Ind{x \ge y} - \Ind{x < y} \right)  + 
y \left( \Ind{y > x} - \Ind{y \le x} \right)  ,
\end{aligned}
\end{equation}
where we have the indicator function defined as
$\Ind{A}=1$ if $A$ is true, and $\Ind{A}=0$ otherwise. 
Let $\C$ be a partition with
$n$ elements. Using the above distance in \eqref{eq:g_def} we have
\begin{equation}
\label{eq:g_ind}
g\left(\C,\C\right) = \dfrac{1}{n^2} \sum_{x \in \C} 
\sum_{y \in \C} 
x \left(
\Ind{x \ge y} + \Ind{y > x} - 
\Ind{x \ge y}-\Ind{x < y} \right) .
\end{equation}
The sum over $y$ can be eliminated since the term in
parenthesis is simply counting the number of elements in $\C$ that satisfy
the conditions of the indicator functions. Assuming
that we first order the data in the partition, obtaining
$\bar{\C} = [ x_j \in \C: x_1 \le x_2 \le \dotsm \le x_{n}]$, we
can write \eqref{eq:g_ind} in the following simple form:
\begin{equation}
\label{eq:g1d}
g\left(\bar{\C}, \bar{\C}\right) = 
\dfrac{2}{n^2} \sum_{\ell=1}^n (2\ell - 1 - n) x_\ell .
\end{equation}
Note that the cost of computing this is $\OO(n)$, and the cost of
sorting the data
is at the most $\OO(n\log n)$.
Assuming that each partition is ordered  $\mathbb{X} = \bigcup_{j=1}^k
\bar{\C}_j$, but notice that the entire data set $\mathbb{X}$ does not 
need to be necessarily ordered, the within energy dispersion
\eqref{eq:within} can be written as
\begin{equation}
\label{eq:w1d}
W\left( \bar{\C}_1,\dotsc,\bar{\C}_k \right) = 
\sum_{j=1}^k \sum_{\ell=1}^{n_j} \dfrac{2\ell - 1 - n_j}{n_j} \, x_\ell.
\end{equation}

For a two-class problem, we can use \eqref{eq:w1d} to cluster data
through a simple algorithm 
as follows. We first order
the entire dataset $\mathbb{X} \to \bar{\mathbb{X}}$. Then 
we compute \eqref{eq:w1d} for each possible split of $\bar{\mathbb{X}}$
and pick the point which gives the minimum value of $W$. 
This procedure is described in Algorithm~\ref{algo1d}. 
Notice that
this method does not require any initialization,
however,
it only works for one-dimensional data with Euclidean distance. The total
complexity of the algorithm is $\OO(n\log n + n^2) = \OO(n^2)$.

\begin{figure}
\begin{algorithm}[H]\vspace{.5em}
\begin{algorithmic}[1]
\INPUT data $\mathbb{X}$
\OUTPUT label matrix $Z$
\STATE sort $\mathbb{X}$ obtaining 
$\bar{\mathbb{X}}= [ x_1,\dotsc,x_n ]$
    \FOR{$j\in [ 1,\dotsc,n ]$}
        \STATE Let $\bar{\C}_1^{(j)} = [x_i: i=1,\dotsc,j]$ and 
                $\bar{\C}_2^{(j)} = [x_i : i=j+1,\dotsc,n]$
        \STATE  
            $W^{(j)} \leftarrow W \big( \bar{\C}_1^{(j)},\bar{\C}_2^{(j)}  
            \big)$ from \eqref{eq:w1d}
    \ENDFOR
    \STATE $j^\star \leftarrow \argmin_j W^{(j)}$ 
    \STATE $Z_{j\bullet} \leftarrow (1,0) $ if $j\le j^\star$, and
           $Z_{j\bullet} \leftarrow (0,1)$ otherwise, for $j=1,\dotsc,n$
\end{algorithmic}
\caption{
\label{algo1d}
Approximate solution to \eqref{eq:minimize} 
for a two-class problem in one dimension. \hspace{\fill}
}
\end{algorithm}
\end{figure}

Assuming the true label matrix $Z$ is available, a direct
measure of how different the estimated matrix $\hat{Z}$ 
is from $Z$, up to label
permutations, is given by
\begin{equation}
\label{eq:accuracy}
\textnormal{accuracy}(\hat{Z}) = \max_\sigma 
\dfrac{1}{n}\sum_{i=1}^n\sum_{j=1}^k \hat{Z}_{i \sigma(j)} Z_{ij}
\end{equation}
where $\sigma$ is a permutation
of the $k$ cluster groups. 
The accuracy is always between $[0,1]$, where
$1$ corresponds to all points correctly clustered, and 
$0$ to all points wrongly clustered.
For a two-class problem with equal
number of points in each cluster, the value $1/2$ correspond
to chance.

Before proposing a more general iterative algorithm to \eqref{eq:qcqp2},
let us consider two simple experiments with equal number of points
in each cluster. 
We keep increasing the number of points in the clusters for each experiment, 
and cluster the data using Algorithm~\ref{algo1d}. 
We also cluster the same data set
with GMM, through EM algorithm, and with $k$-means. In both of these
cases we use the initialization from $k$-means++ \cite{Vassilvitskii} 
and we run the algorithms
few times with different initializations and choose the answer
with best objective function value. We use \eqref{eq:accuracy} to measure
the clustering quality. 
In  Fig.~\ref{fig:1d}a 
we have data from normal distributions,
where we can see that all the three methods
perform closely, with a slight advantage of GMM, as expected, since
it is the right model for the data. However, as shown in Fig~\ref{fig:1d}b,
for lognormal distributions, Algorithm~\ref{algo1d} provides a huge improvement
compared to both GMM and $k$-means which basically cluster at chance.
The zero accuracy values for GMM happened when EM algorithm was unable
to estimate the parameters. These two simple experiments illustrate
how energy statistics based clustering is nonparametric, since it is able
to provide high quality clustering in settings where data comes
from very different distributions.

\begin{figure}
\begin{minipage}{0.49\textwidth}
\includegraphics[width=\textwidth]{normal.pdf}\\[-1em]
(a)
\end{minipage}
\begin{minipage}{0.49\textwidth}
\includegraphics[width=\textwidth]{lognormal.pdf}\\[-1em]
(b)
\end{minipage}
\caption{
\label{fig:1d}
We cluster data using Algorithm~\ref{algo1d} ($1D$-Energy in the plots), 
GMM, and $k$-means. We use
\eqref{eq:accuracy} to evaluate cluster quality. Both clusters
have the same number of points, which are increased in each experiment.
(a) 
$x\sim \tfrac{1}{2}\left( \mathcal{N}(\mu_1,\sigma_1) +
\mathcal{N}(\mu_2,\sigma_2)  \right)$ with 
$\mu_1 = 0$,
$\mu_2 = 5$,
$\sigma_1 = 1$, and
$\sigma_2 = 2$.
(b) 
$x\sim \tfrac{1}{2}\left( e^{\mathcal{N}(\mu_1,\sigma_1)} +
e^{\mathcal{N}(\mu_2,\sigma_2)}  \right)$ with 
$\mu_1 = 0$,
$\mu_2 = -1.5$,
$\sigma_1 = 0.3$, and
$\sigma_2 = 1.5$.
}
\end{figure}




%%%%%%%%%%%%%%%%%%%%%%%%%%%%%%%%%%%%%%%%%%%%%%%%%%%%%%%%%%%%%%%%%%%%%%%%%%%%%%%
\section{Iterative Algorithm for Energy Statistics Clustering}
\label{sec:algo}

In this section we will introduce a new iterative algorithm to find a local
maximizer of the QCQP \eqref{eq:qcqp2}, however, due to 
Proposition~\ref{th:kernel_kmeans} we can also find an approximate
solution by the well-known kernel $k$-means algorithm, which 
for convenience
will also be restated in the present context.
First, let us introduce some base notation.

Consider the optimization problem 
written in the form \eqref{eq:max_prob} as follows:
\begin{equation}
\label{eq:maxQ}
\max_{\{ \C_1,\dotsc,\C_k \}} 
\bigg\{ Q = \sum_{j=1}^k \dfrac{Q_j}{n_j}  \bigg\},
\qquad Q_j = \sum_{x,y\in\C_j} \kk(x,y),
\end{equation}
where $Q_j$ represents an internal energy cost of cluster $\C_j$, and
$Q$ is the total energy cost where each individual cluster cost 
is weighted by the inverse
of the number of its elements. For a data point $x_i$ we denote
its own energy cost
with the entire cluster $\C_\ell$ by
\begin{equation}
\label{eq:costxij}
Q_\ell(x_i) \equiv \sum_{y\in\C_\ell} \kk(x_i, y) = 
G_{i \bullet} \cdot Z_{\bullet \ell},
\end{equation}
where, we recall, $G_{i\bullet}$ ($G_{\bullet i}$) denotes
the $i$th row (column) of matrix $G$.

\subsection{Kernel $\bm{k}$-Means Algorithm}

To optimize kernel $k$-means objective function
\eqref{eq:J}, we remove the global term and define the function
\begin{equation}
\label{eq:Jell}
J^{(\ell)}(x_i) \equiv -\dfrac{2}{n_\ell} Q_\ell(x_i) + \dfrac{1}{n_\ell^2}
Q_\ell,
\end{equation}
which represents a cost depending on point $x_i$ and cluster $\C_\ell$. One
thus assigns  $x_i$ to cluster $\C_{j^\star}$ according
to $j^\star = \argmin_\ell J^{(\ell)}(x_i)$, for $\ell = 1,\dotsc,k$.
This procedure is performed for every data point, and repeated until
convergence, i.e. until no new assignments are made.
The complete algorithm is shown in Algorithm~\ref{kmeans_algo}.
It can be shown that this algorithm converges provided $G$ is positive
semidefinite.
Although our notation looks a little different than the standard
kernel $k$-means found in the literature \cite{Dhillon}, this is precisely
the same algorithm but written in a more concise and explicit way.

Notice that to compute the first term in \eqref{eq:Jell} requires
$\OO(n_\ell)$ operations, and although the second term requires
$\OO(n_\ell^2)$, it only needs to be computed once outside the data
points loop in Algorithm~\ref{kmeans_algo} (step 1).
Therefore, the time complexity Algorithm~\ref{kmeans_algo} is
$\OO(n k \max_\ell n_\ell) = \OO(k n^2)$. For a sparse
Gram matrix $G$ having
$n'$ nonzero elements, this complexity can be further reduced
to $\OO(k n')$. 

\begin{figure}
\begin{algorithm}[H]
\vspace{.5em}
\begin{algorithmic}[1]
    \INPUT number of clusters $k$, Gram matrix $G$, initial label
    matrix $Z = Z_0$
    \OUTPUT label matrix $Z$ 
  \STATE $\bm{q} \leftarrow (Q_1, \dotsc, Q_k)^\top$ 
            have the costs of each cluster, according to \eqref{eq:maxQ}
  \STATE $\bm{n} \leftarrow (n_1,\dotsc,n_k)^\top$ 
        have the number of points in each cluster, obtained 
        from $D = Z^\top Z$
  \REPEAT
    \FOR{ $i=1,\dotsc,n$}
        \STATE let $j$ be such that $x_i \in \C_j$
        \STATE $j^\star \leftarrow \argmin_{\ell} J^{(\ell)}(x_i)$
            according to \eqref{eq:Jell}, for $\ell=1,2,\dots,k$
        \IF{ $j^\star \ne j$} 
            \STATE move $x_i$ to $\C_{j^\star}$: $Z_{ij} \leftarrow 0$ and
            $Z_{ij^\star} \leftarrow 1$
            \STATE update $\bm{n}$: $n_j \leftarrow n_j - 1$ and
                    $n_{j^\star} \leftarrow n_{j^\star} + 1$
            \STATE update $\bm{q}$: $q_j \leftarrow q_j - 2Q_j(x_i)$ and
    $q_{j^\star} \leftarrow q_{j^\star} + 2Q_{j^\star}(x_i)$
    %    \ELSE
    %        \STATE Do nothing;
        \ENDIF
    \ENDFOR
  \UNTIL{convergence}
\end{algorithmic}
\caption{\label{kmeans_algo}
Kernel $k$-means algorithm 
to find an approximate solution to \eqref{eq:qcqp2}.
\hspace{\fill}
}
\end{algorithm}
\end{figure}


\subsection{Energy Cost Algorithm}

Now let us consider a different algorithm, which is based on the change
in the within energy statistics when moving a given data point to
a different cluster.
Suppose we have a data point $x_i \in \mathcal{X}$
which is currently assigned to  cluster $\C_j$, yielding
a total energy cost function \eqref{eq:maxQ} denoted by $Q^{(j)}$.
Let us consider the change in the total energy cost by moving
$x_i$ to cluster $\C_\ell$. 
Denote the new energy cost after moving $x_i$ to $\C_\ell$ by $Q^{(\ell)}$.
It is straightforward to see that
\begin{equation}
\label{eq:changeQ}
\begin{split}
\Delta Q^{j \to \ell}(x_i) &\equiv Q^{(\ell)} - Q^{(j)} \\ 
&= 
\dfrac{1}{n_j - 1}\left[ \dfrac{Q_j}{n_j} - 2 Q_j(x_i) \right]
- \dfrac{1}{n_\ell + 1}\left[ \dfrac{Q_\ell}{n_\ell} - 2 \big(Q_\ell(x_i) + 
\kk(x_i,x_i)\big) 
\right].
\end{split}
\end{equation}
Thus, if $\Delta Q^{j\to \ell}(x_i) > 0$ we get closer to a 
maximum of \eqref{eq:maxQ} by
moving $x_i$ to $\C_\ell$, otherwise we better keep $x_i$ in $\C_j$. Based on
this we propose an algorithm where
the iterates are performed as follows.
We start with an initial configuration for the label matrix $Z$, 
then for each
point $x_i$ 
we compute the cost of moving it to another cluster,
$\Delta Q^{j\to \ell}(x_i)$ for 
$\ell=1,\dots,k$ with $\ell \ne j$. 
We then choose $j^\star = \argmax_\ell \Delta^{j \to \ell}(x_i)$.
If $\Delta Q^{j \to j^\star}(x_i) > 0$ 
we move $x_i$ to cluster $\C_{j^\star}$, otherwise 
we keep $x_i$ in its original cluster $\C_j$. We update $Z$ accordingly.
The process is repeated
until convergence, i.e. until no points are assigned to new clusters. 
This whole procedure is described in Algorithm~\ref{algo}.
Note that \eqref{eq:changeQ} assures that the objective function is
monotonically increasing at each iteration.

\begin{figure}
\begin{algorithm}[H]
\vspace{.5em}
\begin{algorithmic}[1]
    \INPUT number of clusters $k$, Gram matrix $G$, 
                initial label matrix $Z=Z_0$
    \OUTPUT label matrix $Z$
  \STATE $\bm{q} \leftarrow (Q_1, \dotsc, Q_k)^\top$ 
            have the energy costs of each cluster, according to \eqref{eq:maxQ}
  \STATE $\bm{n} \leftarrow (n_1,\dotsc,n_k)^\top$ have the number of points 
        in each cluster, obtained from $D=Z^\top Z$
  \REPEAT
    \FOR{ $i=1,\dotsc,n$}
        \STATE let $j$ be such that $x_i \in \C_j$
        \STATE $j^\star \leftarrow \argmax_{\ell} \Delta Q^{j\to \ell}(x_i)$, 
            for $\ell=1,2,\dots,k$ and $\ell \ne j$ \label{stepmove}
        \IF{ $\Delta Q^{j \to j^\star}(x_i) > 0$ }
            \STATE move $x_i$ to $\C_{j^\star}$: $Z_{ij} \leftarrow 0$ and 
            $Z_{ij^\star} \leftarrow 1$
            \STATE update $\bm{n}$: $n_j \leftarrow n_j - 1$ and
                    $n_{j^\star} \leftarrow n_{j^\star} + 1$
            \STATE update $\bm{q}$: $q_j \leftarrow q_j - 2Q_j(x_i)$ and
    $q_{j^\star} \leftarrow q_{j^\star} + 2\left(Q_{j^\star}(x_i)+
    G_{ii}\right)$
    %    \ELSE
    %        \STATE Do nothing;
        \ENDIF
    \ENDFOR
  \UNTIL{convergence}
\end{algorithmic}
\caption{\label{algo}
Energy cost algorithm to find an approximate solution to \eqref{eq:qcqp2}.
\hspace{\fill}
}
\end{algorithm}
\end{figure}

Notice that computing $G$ requires $\OO( D n^2)$ operations, where 
$D$ is the dimension of each data point and $n$ is the data size. However,
both previous algorithms assume that $G$ is given. There are more efficient
methods to compute $G$, specially if it is sparse. We will not consider
this further, and just assume that $G$ is given.
The computation of each cluster cost
$Q_j$ has complexity $\OO(n_j^2)$, and overall to compute $\bm{q}$
we have $\OO(n_1^2+\dots + n_k^2) = \OO(k \max_j n_j^2)$. 
These operations, however, only need to be performed a single time. Now for
each point $x_i$ we need to compute $Q_j(x_i)$ once, which is
$\OO(n_j)$, and we need to compute $Q_\ell(x_i)$ for each $\ell\ne j$. 
The cost of computing 
\eqref{eq:costxij} is $\OO(n_j)$, thus the cost of step~$8$ in
Algorithm~\ref{algo} is $\OO(k \max_j n_j)$ for $j=1,\dotsc,k$.
For the 
entire dataset this gives a time-complexity
of $\OO(n k  \max_j n_j) =\OO(k n^2)$. This is the same cost as
in kernel $k$-means, Algorithm~\ref{kmeans_algo}. Again, if $G$ is sparse
this can be reduced to $\OO(k n')$, where $n'$ is the number of nonzero
entries of $G$.


%%%%%%%%%%%%%%%%%%%%%%%%%%%%%%%%%%%%%%%%%%%%%%%%%%%%%%%%%%%%%%%%%%%%%%%%%%%%%%%
\section{Numerical Experiments}
\label{sec:numerics}

In the experiments below we fix the semimetric 
according to the traditional energy distance \eqref{eq:energy}, and
the point $x_0=0$ is chosen in the associated kernel  
\eqref{eq:kernel_semimetric}. We thus have
\begin{equation}
\label{eq:standard_metric}
\rho(x,y) = \| x-y\|, \qquad \kk(x,y) = 
\tfrac{1}{2}\left( \| x \| + \| y \| - \| x-y \| \right).
\end{equation}
We will consider other semimetrics/kernels as well, 
but the above will be considered the
standard kernel for energy statistics and will always be present in every
experiment as a reference. Notice that this is a convention, 
we could have chosen
any other semimetric as the standard.
One of the main goals of the following 
experiments is to compare Algorithm~\ref{algo} to
kernel $k$-means algorithm, described in Algorithm~\ref{kmeans_algo}. Thus,
for every kernel used in Algorithm~\ref{algo}, we also use the same 
kernel in Algorithm~\ref{kmeans_algo}. Another goal is to compare
Algorithm~\ref{algo} with $k$-means and GMM (through expectation maximization
algorithm), as these are the most used
clustering algorithms in practice.
Since for synthetic data the true labels are available,
our measure of clustering quality
will be \eqref{eq:accuracy}. Moreover, for all algorithms, we always
choose the initialization from $k$-means++ \cite{Vassilvitskii}.

We first consider clustering in high dimensions and analyze 
how the algorithms degrade as the number of dimensions increase, while
keeping the number of points in each cluster fixed. The Bayes error
is also kept fixed as ambient dimension increases.
In Figure~\ref{fig:gauss}a we have data generated from $D$-variate normal 
distributions as follows: 
\begin{equation}
\label{eq:gauss1}
\begin{split}
&x \sim \tfrac{1}{2}\left[ 
\mathcal{N}(\mu_1,\Sigma_1) + \mathcal{N}(\mu_2, \Sigma_2)\right], \\
&\mu_1 = (\underbrace{0,\dotsc,0}_{\times D})^\top , \quad
\mu_2 = 0.7 \times (\underbrace{1,\dots,1}_{\times 10},
\underbrace{0,\dots,0}_{\times (D-10)})^\top, \quad
\Sigma_1 = \Sigma_2 = I_D.
\end{split}
\end{equation}
We only keep signal in
in the first $10$
dimensions of $\mu_2$, and keep increasing the ambient dimension $D$. For each
$D$, we perform $100$ experiments, obtaining the clustering accuracy
for each algorithm.
We can see that
GMM is not able to estimate the covariance matrix 
when the number
of dimensions exceeds the number of points in each cluster, so it gives
zero accuracy for $D \gtrsim 100$.
In Figure~\ref{fig:gauss}b we have the same type of experiment but 
with 
\begin{equation}
\label{eq:gauss2}
\begin{split}
&x \sim \tfrac{1}{2}\left[ 
\mathcal{N}(\mu_1,\Sigma_1) + \mathcal{N}(\mu_2, \Sigma_2)\right], \\
&\mu_1 = (\underbrace{0,\dotsc,0}_{\times D})^\top , \,
\mu_2 = 0.7 \times (\underbrace{1,\dots,1}_{\times 10},
\underbrace{0,\dots,0}_{\times (D-10)})^\top, \,
\Sigma_1 = I_D, \, 
\Sigma_2 = \left( \begin{smallmatrix} \tfrac{1}{2} I_{10} & 0 \\ 0 & I_{D-10}
\end{smallmatrix}\right). \quad
\end{split}
\end{equation}
Therefore, for both experiments shown in Figure~\ref{fig:gauss}
we can see a better performance of Algorithm~\ref{algo} compared
to the other ones, in particular compared to kernel $k$-means algorithm,
where we recall that both
aim at optimizing the same problem (see Proposition~\ref{th:kernel_kmeans}). 
Also, notice that $k$-means
and GMM are consistently the right model for this dataset, so it is hard
to perform better than these algorithms in this current setting.
Notice that Algorithm~\ref{algo} is more robust as the ambient
dimension increases.

\begin{figure}
\begin{minipage}{0.49\textwidth}
\centering
\includegraphics[width=1\textwidth]{gauss_dim.pdf}\\[-1.0em]
(a)
\end{minipage}
\begin{minipage}{0.49\textwidth}
\centering
\includegraphics[width=1\textwidth]{gauss_cov.pdf}\\[-1.0em]
(b)
\end{minipage}
\caption{
\label{fig:gauss}
Effect of increasing the
ambient dimension while keeping Bayes error fixed, for 
two clusters with normally distributed data with $100$ points in each
cluster. 
(a) 
We increase $D$ as described in \eqref{eq:gauss1}. The blue
line correspond to Algorithm~\ref{algo}, while
the magenta line 
corresponds to kernel $k$-means, Algorithm~\ref{kmeans_algo}.
(b) The same but with data following \eqref{eq:gauss2}.
One notices that
Algorithm~\ref{algo} is more robust than the other ones.
}
\end{figure}

In Figure~\ref{fig:unbalanced} we consider the effect of having 
unbalanced clusters. We generate data as
\begin{equation}
\label{eq:gauss3}
\begin{split}
&x \sim  
\dfrac{n_1}{N} \mathcal{N}(\mu_1,\Sigma_1) + 
\dfrac{n_2}{N} \mathcal{N}(\mu_2, \Sigma_2), 
\quad \mu_1 = (0,0,0,0)^\top , \,
\mu_2 = 1.5\times (1,1,0,0)^\top, \\
&
\Sigma_1 = I_4, \quad
\Sigma_2 = \left( 
\begin{smallmatrix} 
1/2 & 0 & 0 & 0\\
0 & 1/2 & 0 & 0 \\
0 & 0 & 1 & 0 \\
0 & 0 & 0 & 1 
\end{smallmatrix}\right), \quad
n_1 = N - m, \quad  n_2 = N + m, \quad N=200.
\end{split}
\end{equation}
We then increase $m$, i.e. we make the clusters progressively more unbalanced.
We generate $100$ experiments for each $m$, and plot the clustering accuracy
versus $m$.
As expected, GMM
works better than the other algorithms in the case of unbalanced clusters. 
This is mostly due to its soft assignments.
We can see that the other methods based on hard assignments degrade similarly,
and  more rapidly than GMM. This indicates that a fuzzy version of
energy statistics clustering should compensate for this effect.

\begin{figure}
\centering
\includegraphics[width=0.5\textwidth]{gauss_pi.pdf}
\caption{
\label{fig:unbalanced}
Previous algorithms for unbalanced clusters, 
according to \eqref{eq:gauss3}.}
\end{figure}

Now, besides \eqref{eq:standard_metric} we consider two other semimetrics:
\begin{align}
\rho_{1/2}(x,y) &= \| x-y \|^{1/2}, & \kk(x,y) &= \tfrac{1}{2} \left( 
\| x \|^{1/2} + \| y \|^{1/2} 
- \| x-y \|^{1/2} \right), \label{eq:rhohalf}\\
\rho_{e}(x,y) &= 2 - 2 e^{-\| x- y\|/2}, & \kk(x,y) &= e^{-\| x-y\|/2}.
\label{eq:rhoe}
\end{align}
In Figure~\ref{fig:consist}a we have data in $20$ dimensions distributed as
\begin{equation}
\label{eq:20gauss}
\begin{split}
x &\sim \tfrac{1}{2}\left[ \mathcal{N}(\mu_1,\Sigma_1) + \mathcal{N}(\mu_2,
\Sigma_2) \right], \\
\mu_1 &= (\underbrace{0,\dotsc,0}_{\times 20})^\top ,\quad
\mu_2 = \tfrac{1}{2} 
(\underbrace{1,\dotsc,1}_{5},\underbrace{0,\dotsc,0}_{15})^\top, \quad
\Sigma_1 = \tfrac{1}{2} I_{20},  \quad
\Sigma_2 = I_{20}.
\end{split}
\end{equation}
We increase the number of points in each cluster and show the clustering
accuracy with different algorithms. The new semimetrics \eqref{eq:rhohalf}
and $\eqref{eq:rhoe}$ are indicated in the legend. One can see that
Algorithm~\ref{algo} performs better than all the other ones, and in 
particular \eqref{eq:rhoe} provides better results.
As the number of datapoints 
get large enough, GMM starts to be as accurate as clustering
based on energy statistics, as it should since it is 
consistent model to the data. In Figure~\ref{fig:consist}b, however, we
use the same parameters as in \eqref{eq:20gauss} but now with data
log-normally distributed:
\begin{equation}
\label{eq:20loggauss}
x \sim \tfrac{1}{2}\left[ e^{\mathcal{N}(\mu_1,\Sigma_1)}
+ e^{\mathcal{N}(\mu_2, \Sigma_2)}\right].
\end{equation}
We see that clustering based on energy statistics still performs accurately
for this kind of data, while $k$-means works a little bit better than chance,
and GMM is not even able to estimate the parameters. Again, \eqref{eq:rhoe}
provides slightly better results than \eqref{eq:standard_metric} or
\eqref{eq:rhohalf}. Notice also that Algorithm~\ref{algo} performs better
than Algorithm~\ref{kmeans_algo}. Both experiments in Figure~\ref{fig:consist}
shows that energy statistics clustering is nonparametric, since it is able
to cluster data coming from very different distributions.

\begin{figure}
\begin{minipage}{0.49\textwidth}
\centering
\includegraphics[width=1\textwidth]{gauss.pdf}\\[-1.0em]
(a)
\end{minipage}
\begin{minipage}{0.49\textwidth}
\centering
\includegraphics[width=1\textwidth]{loggauss.pdf}\\[-1.0em]
(b)
\end{minipage}
\caption{
\label{fig:consist}
(a) Data normally distributed as in 
\eqref{eq:20gauss}. We increase the number of points in each cluster to
illustrate the statistical consistency of the algorithms.
(b) The same experiment but for data following \eqref{eq:20loggauss}.
In both experiments, for each case we run every algorithm $100$ times and show
the average results. One can see the better performance of energy statistics
clustering, Algorithm~\ref{algo}, and in particular by using the semimetric
\eqref{eq:rhoe}. These two figures illustrate that energy statistics clustering
is nonparametric since it works well for very different distributions.
}
\end{figure}


%%%%%%%%%%%%%%%%%%%%%%%%%%%%%%%%%%%%%%%%%%%%%%%%%%%%%%%%%%%%%%%%%%%%%%%%%%%%%%%
\section{Conclusion}
\label{sec:conclusion}

In this paper we have considered clustering from the perspective of energy
statistics, which provides a nonparametric test for equality of distributions.
Based on this, we showed that the clustering problem reduces to a quadratically
constrained optimization problem (QCQP), 
as described in Proposition~\ref{th:qcqp2}.
Moreover, we showed that clustering based on energy statistics is equivalent
to kernel $k$-means optimization problem, once the kernel is fixed; see
Proposition~\ref{th:kernel_kmeans}. Our results imply that kernel $k$-means
approach to clustering is actually a consequence of energy statistics
theory, and thus place this method into a principled statistical basis.
As already known \cite{Dhillon}, this approach is related to spectral
clustering, and graph partitioning problems. Therefore, all these problems 
may be seen as arising naturally from energy statistics clustering.
It is important to mention that energy statistics clustering, as formulated
here, is valid for
arbitrary metric spaces of negative type, and makes no assumptions about
the distribution of the data. Moreover, it does not rely on the concept of a
cluster mean, even implicitly.

We also proposed Algorithm~\ref{algo} as an alternative to the well-known
kernel $k$-means algorithm (see Algorithm~\ref{kmeans_algo}), 
where both have the same time
complexity. The numerical results show that Algorithm~\ref{algo}
might provide better clustering accuracy and is more robust 
than kernel $k$-means algorithm. Since there exists a huge literature about
kernel $k$-means, and approximation methods to make it faster, with
applications to several artificial and real data, 
we limited ourselves to analyze few but carefully designed experiments, which
illustrates the advantages of Algorithm~\ref{algo}.


\bigskip

%\subsection*{Acknowledgements}
%We thank \ldots


\bibliographystyle{unsrt}
\bibliography{biblio.bib}



\end{document}
